\documentclass[a5paper, 10pt]{article}

% Текст
\usepackage[utf8]{inputenc} % UTF-8 кодировка
\usepackage[russian]{babel} % Русский язык
\usepackage{indentfirst} % красная строка в первом параграфе в главе
% Отображение страниц
\usepackage{geometry} % размеры листа и отступов
\usepackage{listings}
\usepackage{color}

\geometry{
	left=12mm,
	top=25mm,
	right=15mm,
	bottom=17mm,
	marginparsep=0mm,
	marginparwidth=0mm,
	headheight=10mm,
	headsep=7mm,
	nofoot}
\usepackage{afterpage,fancyhdr} % настройка колонтитулов
\pagestyle{fancy}
\fancypagestyle{style}{ % создание нового стиля style
	\fancyhf{} % очистка колонтитулов
	\fancyhead[LO, RE]{Типовой расчет № 1} % название документа наверху
	\fancyhead[RO, LE]{ } % название section наверху
	\fancyfoot[RO, LE]{\thepage} % номер страницы справа внизу на нечетных и слева внизу на четных
	\renewcommand{\headrulewidth}{0.25pt} % толщина линии сверху
	\renewcommand{\footrulewidth}{0pt} % толцина линии снизу
}
\fancypagestyle{plain}{ % создание нового стиля plain -- полностью пустого
	\fancyhf{}
	\renewcommand{\headrulewidth}{0pt}
}
\fancypagestyle{title}{ % создание нового стиля title -- для титульной страницы
	\fancyhf{}
	\fancyhead[C]{{\footnotesize
			Министерство образования и науки Российской Федерации\\
			Федеральное государственное автономное образовательное учреждение высшего образования
	}}
	\fancyfoot[C]{{\large 
			Санкт-Петербург, 2023-2024
	}}
	\renewcommand{\headrulewidth}{0pt}
}

% Математика
\usepackage{amsmath, amsfonts, amssymb, amsthm} % Набор пакетов для математических текстов
%\usepackage{dmvnbase} % мехматовский пакет latex-сокращений
\usepackage{cancel} % зачеркивание для сокращений
% Рисунки и фигуры
\usepackage[pdftex]{graphicx} % вставка рисунков
\usepackage{wrapfig, subcaption} % вставка фигур, обтекая текст
\usepackage{caption} % для настройки подписей
\captionsetup{figurewithin=none,labelsep=period, font={small,it}} % настройка подписей к рисункам
% Рисование
\usepackage{tikz} % рисование
\usepackage{circuitikz}
\usepackage{pgfplots} % графики
% Таблицы
\usepackage{multirow} % объединение строк
\usepackage{multicol} % объединение столбцов
% Остальное
\usepackage[unicode, pdftex]{hyperref} % гиперссылки
\usepackage{enumitem} % нормальное оформление списков
\setlist{itemsep=0.15cm,topsep=0.15cm,parsep=1pt} % настройки списков
% Теоремы, леммы, определения...
\theoremstyle{definition}
\newtheorem{Def}{Определение}
\newtheorem*{Axiom}{Аксиома}
\theoremstyle{plain}
\newtheorem{Th}{Теорема}
\newtheorem{Lem}{Лемма}
\newtheorem{Cor}{Следствие}
\newtheorem{Ex}{Пример}
\theoremstyle{remark}
\newtheorem*{Note}{Замечание}
\newtheorem*{Solution}{Решение}
\newtheorem*{Proof}{Доказательство}
% Свои команды
\newcommand{\comb}[1]{\left[\hspace{-4pt}\begin{array}{l}#1\end{array}\right.\hspace{-5pt} } % совокупность уравнений
% Титульный лист
\usepackage{csvsimple-l3}
\newcommand*{\titlePage}{
	\thispagestyle{title}
	\begingroup
	\begin{center}
		%		{\footnotesize
			%			Министерство образования и науки Российской Федерации\\
			%			Федеральное государственное автономное образовательное учреждение высшего образования
			%		}
		%		
		\vspace*{6ex}
		
		{\small
			САНКТ-ПЕТЕРБУРГСКИЙ НАЦИОНАЛЬНЫЙ ИССЛЕДОВАТЕЛЬСКИЙ УНИВЕРСИТЕТ ИТМО	
		}
		
		\vspace*{2ex}
		
		{\normalsize
			Факультет систем управления и робототехники
		}
		
		\vspace*{15ex}
		
		{\Large \bfseries 
			Типовой расчет № 1
		}
\vspace*{2ex}
	{\Large \bfseries 
			
"Функции нескольких переменных "
		}
\vspace*{2ex}
		
		{\normalsize
			по дисциплине Математический анализ
		}

	\end{center}
	\vspace*{20ex}
	\begin{flushright}
		{\large 
			\underline{Выполнила}: студентка гр. \textbf{R3238}\\
			\begin{flushright}
				\textbf{Нечаева А. А.}\\
			\end{flushright}
		}
		
		\vspace*{5ex}
		
		{\large 
			\underline{Преподаватель}: \textit{Бойцев Антон Александрович}
		}
	\end{flushright}	
	\newpage
	\setcounter{page}{1}
	\endgroup}

\begin{document}
	\titlePage
	\pagestyle{style}
\newpage

\section{задание.}
\textit{Найти частные производные данной функции $f(x, y)$ в точке $(0, 0)$. Выяснить, является ли функция дифференцируемой в точке $(0, 0)$. Найти её дифференциал. Пункт 2.}
\begin{equation}
f(x, y) = y + \cos \sqrt[3]{x^2 + y^2}
\end{equation}
\subsection{Частные производные}

\begin{equation}
\frac{\partial f}{\partial x} (x_0, y_0) = \lim_{\Delta x \to 0} \frac{\Delta_x f}{\Delta x} = \lim_{\Delta x \to 0} \frac{f ( x_0 + \Delta x, y_0) - f(x_0, y_0)}{\Delta x}
\end{equation}
\\

Производная по $x$ в точке $(0, 0)$:
\begin{multline}
\frac{\partial f}{\partial x} (0, 0) = \lim_{\Delta x \to 0} \frac{\cos \sqrt[3]{( \Delta x)^2} - \cos 0}{\Delta x} = 
-2 \lim_{\Delta x \to 0} \frac{\sin^2  \frac{ \sqrt[3]{( \Delta x)^2}}{2}}{\Delta x} =\\
= -2 \lim_{\Delta x \to 0} \frac{\sin^2  \frac{ \sqrt[3]{( \Delta x)^2}}{2} \sqrt[3]{ \Delta x}}{4 \left(\frac{ \sqrt[3]{( \Delta x)^2}}{2} \right)^2} = 0
\end{multline}


Производная по $y$ в точке $(0, 0)$:
\begin{multline}
\frac{\partial f}{\partial y} (0, 0) = \lim_{\Delta y \to 0} \frac{\Delta y + \cos \sqrt[3]{( \Delta y)^2} - \cos 0}{\Delta y} = 
\lim_{\Delta y \to 0} \left(1 +  \frac{\cos \sqrt[3]{( \Delta y)^2} - 1}{\Delta y} \right) =\\
= 1 - 2 \lim_{\Delta y \to 0} \frac{\sin^2  \frac{ \sqrt[3]{( \Delta y)^2}}{2}}{\Delta y} 
=1 -2 \lim_{\Delta y \to 0} \frac{\sin^2  \frac{ \sqrt[3]{( \Delta y)^2}}{2} \sqrt[3]{ \Delta y}}{4 \left(\frac{ \sqrt[3]{( \Delta y)^2}}{2} \right)^2} = 1
\end{multline}


\subsection{Дифференцируемость в точке  $(0, 0)$}


\subsection{Дифференциал}

\begin{equation}
df = \frac{\partial f}{\partial x} dx + \frac{\partial f}{\partial y} dy
\end{equation}
Найдем частные производные:

\begin{equation}
\frac{\partial f}{\partial x} = \left( y + \cos \sqrt[3]{x^2 + y^2} \right)'_x = -\frac{2x}{3} \frac{\sin \sqrt[3]{x^2 + y^2}}{\sqrt[3]{ \left( x^2 + y^2 \right)^2}}
\end{equation}

\begin{equation}
\frac{\partial f}{\partial y} = \left( y + \cos \sqrt[3]{x^2 + y^2} \right)'_y = 1 -\frac{2y}{3} \frac{\sin \sqrt[3]{x^2 + y^2}}{\sqrt[3]{ \left( x^2 + y^2 \right)^2}}
\end{equation}
Терерь запишем полный дифференциал:

\begin{equation}
df =  -\frac{2x}{3} \frac{\sin \sqrt[3]{x^2 + y^2}}{\sqrt[3]{ \left( x^2 + y^2 \right)^2}} dx + \left( 1 -\frac{2y}{3} \frac{\sin \sqrt[3]{x^2 + y^2}}{\sqrt[3]{ \left( x^2 + y^2 \right)^2}}  \right) dy
\end{equation}

\newpage

\section{задание}
\textit{Найти производную данной функции в направлении данного вектора в заданной точке $M$. Пункт 8.}

\begin{equation}
f(x, y, z) = exp(x + 2xy + 3xyz)
\end{equation}
по направлению внутренней нормали к поверхности $x^2 + y^2 + z^2 + 2z = 1$, $M \left( \frac{1}{2}, \frac{\sqrt{3}}{2}, 0 \right)$\\
\\
Найдем уравнение касательной плоскости к поверхности $x^2 + y^2 + z^2 + 2z = 1$ в точке $M \left( \frac{1}{2}, \frac{\sqrt{3}}{2}, 0 \right)$:

\begin{equation}
F'_x \left( M \right) \cdot (x - x_0) + F'_y \left( M \right) \cdot (y - y_0) + F'_z \left( M \right) \cdot (z - z_0) = 0
\end{equation}

\begin{equation}
F'_x = \left(  x^2 + y^2 + z^2 + 2z - 1 \right)'_x = 2x
\end{equation}

\begin{equation}
F'_y = \left(  x^2 + y^2 + z^2 + 2z - 1 \right)'_y = 2y
\end{equation}

\begin{equation}
F'_z = \left(  x^2 + y^2 + z^2 + 2z - 1 \right)'_z = 2z + 2
\end{equation}

\begin{equation}
F'_x  \left( M \right) = 2 \frac{1}{2} = 1
\end{equation}

\begin{equation}
F'_y  \left( M \right) = 2 \frac{\sqrt{3}}{2} = \sqrt{3}
\end{equation}

\begin{equation}
F'_z  \left( M \right) = 0
\end{equation}




\newpage

\section{задание}
\textit{Произвести указанную замену в данном дифференциальном уравнении. Решить полученное дифференциальное уравнение в новых переменных. Показать, что найденное решение (в исходных переменных) удовлетворяет исходному уравнению. Пункт 3.}\\
\\
$u$ и $v$ -- новые независимые переменные, $w$ -- новая функция. $u=x+y, \, v = x - y, \, w+z = xy,$

\begin{equation}
\frac{\partial^2 z}{\partial x^2} + 2 \frac{\partial^2 z}{\partial x \partial y} + \frac{\partial^2 z}{\partial y^2} = 0
\end{equation}
 $z=z(x, y)$, $w=w(u, v)$, $u = u(x, y)$, $v = v(x, y)$

Для начала найдем все производные второго порядка функции $w$ по переменным $x$ и $y$:
\begin{equation}
w'_x = w'_u \cdot u'_x + w'_v \cdot v'_x = w'_u + w'_v
\end{equation}

% for x
\begin{equation}
w''_{ux}  = w''_{uu} \cdot u'_x + w''_{uv} \cdot v'_x = w''_{uu} + w''_{uv}
\end{equation}

\begin{equation}
w''_{vx}  = w''_{vu} \cdot u'_x + w''_{vv} \cdot v'_x = w''_{vu} + w''_{vv}
\end{equation}

Вторая производная по  $x$:
\begin{equation}
w''_{xx} = w''_{uu} + 2w''_{uv} + w''_{vv}
\end{equation}


\begin{equation}
w'_y = w'_u \cdot u'_y + w'_v \cdot v'_y = w'_u - w'_v
\end{equation}

\begin{equation}
w''_{uy}  = w''_{uu} \cdot u'_y + w''_{uv} \cdot v'_y = w''_{uu} - w''_{uv}
\end{equation}

\begin{equation}
w''_{vy}  = w''_{vu} \cdot u'_y + w''_{vv} \cdot v'_y = w''_{vu} - w''_{vv}
\end{equation}

 Вторая производная по $y$:
\begin{equation}
w''_{yy} = w''_{uu} - 2w''_{uv} + w''_{vv}
\end{equation}

Вторая производная по  $x$ и $y$:

\begin{equation}
w''_{yx} = w''_{xy} =  w''_{uu} -  w''_{vv}
\end{equation}
\\
Теперь выразим $z''_{xx}$,  $z''_{xy}$ и $z''_{yy}$ из $w + z = xy$:

\end{document}













