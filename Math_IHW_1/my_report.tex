\documentclass[a5paper, 10pt]{article}

% Текст
\usepackage[utf8]{inputenc} % UTF-8 кодировка
\usepackage[russian]{babel} % Русский язык
\usepackage{indentfirst} % красная строка в первом параграфе в главе
% Отображение страниц
\usepackage{geometry} % размеры листа и отступов
\usepackage{listings}
\usepackage{color}

\geometry{
	left=12mm,
	top=25mm,
	right=15mm,
	bottom=17mm,
	marginparsep=0mm,
	marginparwidth=0mm,
	headheight=10mm,
	headsep=7mm,
	nofoot}
\usepackage{afterpage,fancyhdr} % настройка колонтитулов
\pagestyle{fancy}
\fancypagestyle{style}{ % создание нового стиля style
	\fancyhf{} % очистка колонтитулов
	\fancyhead[LO, RE]{Типовой расчет № 1} % название документа наверху
	\fancyhead[RO, LE]{ } % название section наверху
	\fancyfoot[RO, LE]{\thepage} % номер страницы справа внизу на нечетных и слева внизу на четных
	\renewcommand{\headrulewidth}{0.25pt} % толщина линии сверху
	\renewcommand{\footrulewidth}{0pt} % толцина линии снизу
}
\fancypagestyle{plain}{ % создание нового стиля plain -- полностью пустого
	\fancyhf{}
	\renewcommand{\headrulewidth}{0pt}
}
\fancypagestyle{title}{ % создание нового стиля title -- для титульной страницы
	\fancyhf{}
	\fancyhead[C]{{\footnotesize
			Министерство образования и науки Российской Федерации\\
			Федеральное государственное автономное образовательное учреждение высшего образования
	}}
	\fancyfoot[C]{{\large 
			Санкт-Петербург, 2023-2024
	}}
	\renewcommand{\headrulewidth}{0pt}
}

% Математика
\usepackage{amsmath, amsfonts, amssymb, amsthm} % Набор пакетов для математических текстов
%\usepackage{dmvnbase} % мехматовский пакет latex-сокращений
\usepackage{cancel} % зачеркивание для сокращений
% Рисунки и фигуры
\usepackage[pdftex]{graphicx} % вставка рисунков
\usepackage{wrapfig, subcaption} % вставка фигур, обтекая текст
\usepackage{caption} % для настройки подписей
\captionsetup{figurewithin=none,labelsep=period, font={small,it}} % настройка подписей к рисункам
% Рисование
\usepackage{tikz} % рисование
\usepackage{circuitikz}
\usepackage{pgfplots} % графики
% Таблицы
\usepackage{multirow} % объединение строк
\usepackage{multicol} % объединение столбцов
% Остальное
\usepackage[unicode, pdftex]{hyperref} % гиперссылки
\usepackage{enumitem} % нормальное оформление списков
\setlist{itemsep=0.15cm,topsep=0.15cm,parsep=1pt} % настройки списков
% Теоремы, леммы, определения...
\theoremstyle{definition}
\newtheorem{Def}{Определение}
\newtheorem*{Axiom}{Аксиома}
\theoremstyle{plain}
\newtheorem{Th}{Теорема}
\newtheorem{Lem}{Лемма}
\newtheorem{Cor}{Следствие}
\newtheorem{Ex}{Пример}
\theoremstyle{remark}
\newtheorem*{Note}{Замечание}
\newtheorem*{Solution}{Решение}
\newtheorem*{Proof}{Доказательство}
% Свои команды
\newcommand{\comb}[1]{\left[\hspace{-4pt}\begin{array}{l}#1\end{array}\right.\hspace{-5pt} } % совокупность уравнений
% Титульный лист
\usepackage{csvsimple-l3}
\newcommand*{\titlePage}{
	\thispagestyle{title}
	\begingroup
	\begin{center}
		%		{\footnotesize
			%			Министерство образования и науки Российской Федерации\\
			%			Федеральное государственное автономное образовательное учреждение высшего образования
			%		}
		%		
		\vspace*{6ex}
		
		{\small
			САНКТ-ПЕТЕРБУРГСКИЙ НАЦИОНАЛЬНЫЙ ИССЛЕДОВАТЕЛЬСКИЙ УНИВЕРСИТЕТ ИТМО	
		}
		
		\vspace*{2ex}
		
		{\normalsize
			Факультет систем управления и робототехники
		}
		
		\vspace*{15ex}
		
		{\Large \bfseries 
			Типовой расчет № 1
		}
\vspace*{2ex}
	{\Large \bfseries 
			
"Функции нескольких переменных "
		}
\vspace*{2ex}
		
		{\normalsize
			по дисциплине Математический анализ
		}

	\end{center}
	\vspace*{20ex}
	\begin{flushright}
		{\large 
			\underline{Выполнила}: студентка гр. \textbf{R3238}\\
			\begin{flushright}
				\textbf{Нечаева А. А.}\\
			\end{flushright}
		}
		
		\vspace*{5ex}
		
		{\large 
			\underline{Преподаватель}: \textit{Бойцев Антон Александрович}
		}
	\end{flushright}	
	\newpage
	\setcounter{page}{1}
	\endgroup}

\begin{document}
	\titlePage
	\pagestyle{style}
\newpage

\section{задание.}
\textit{Найти частные производные данной функции $f(x, y)$ в точке $(0, 0)$. Выяснить, является ли функция дифференцируемой в точке $(0, 0)$. Найти её дифференциал. Пункт 2.}
\begin{equation}
f(x, y) = y + \cos \sqrt[3]{x^2 + y^2}
\end{equation}
\subsection{Частные производные}

\begin{equation}
\frac{\partial f}{\partial x} (x_0, y_0) = \lim_{\Delta x \to 0} \frac{\Delta_x f}{\Delta x} = \lim_{\Delta x \to 0} \frac{f ( x_0 + \Delta x, y_0) - f(x_0, y_0)}{\Delta x}
\end{equation}
\\

Производная по $x$ в точке $(0, 0)$:
\begin{multline}
\frac{\partial f}{\partial x} (0, 0) = \lim_{\Delta x \to 0} \frac{\cos \sqrt[3]{( \Delta x)^2} - \cos 0}{\Delta x} = 
-2 \lim_{\Delta x \to 0} \frac{\sin^2  \frac{ \sqrt[3]{( \Delta x)^2}}{2}}{\Delta x} =\\
= -2 \lim_{\Delta x \to 0} \frac{\sin^2  \frac{ \sqrt[3]{( \Delta x)^2}}{2} \sqrt[3]{ \Delta x}}{4 \left(\frac{ \sqrt[3]{( \Delta x)^2}}{2} \right)^2} = 0
\end{multline}


Производная по $y$ в точке $(0, 0)$:
\begin{multline}
\frac{\partial f}{\partial y} (0, 0) = \lim_{\Delta y \to 0} \frac{\Delta y + \cos \sqrt[3]{( \Delta y)^2} - \cos 0}{\Delta y} = 
\lim_{\Delta y \to 0} \left(1 +  \frac{\cos \sqrt[3]{( \Delta y)^2} - 1}{\Delta y} \right) =\\
= 1 - 2 \lim_{\Delta y \to 0} \frac{\sin^2  \frac{ \sqrt[3]{( \Delta y)^2}}{2}}{\Delta y} 
=1 -2 \lim_{\Delta y \to 0} \frac{\sin^2  \frac{ \sqrt[3]{( \Delta y)^2}}{2} \sqrt[3]{ \Delta y}}{4 \left(\frac{ \sqrt[3]{( \Delta y)^2}}{2} \right)^2} = 1
\end{multline}


\subsection{Дифференцируемость в точке  $(0, 0)$}
Функция будет дифференцируема в точке $\left(0;0 \right)$, если для ее приращения в этой точке выполняется равенство:
\begin{equation}
\Delta f (0;0) = \frac{\partial f}{\partial x} (0;0) \cdot \Delta x +  \frac{\partial f}{\partial y} (0;0) \cdot \Delta y + o(\rho)\, ,
\end{equation}
где $\rho = \sqrt{\Delta x^2 + \Delta y^2}$.\\
Запишем формулу, подставив в нее выше найденные производные:
\begin{equation}
\Delta f (0;0) = 0 \cdot \Delta x +  1 \cdot \Delta y + o( \sqrt{\Delta x^2 + \Delta y^2}) = \Delta y + o( \sqrt{\Delta x^2 + \Delta y^2})
\end{equation}
\textit{\textbf{Ответ:}} функция дифференцируема в точке $(0, 0)$.

\subsection{Дифференциал}

\begin{equation}
df = \frac{\partial f}{\partial x} dx + \frac{\partial f}{\partial y} dy
\end{equation}
Найдем частные производные:

\begin{equation}
\frac{\partial f}{\partial x} = \left( y + \cos \sqrt[3]{x^2 + y^2} \right)'_x = -\frac{2x}{3} \frac{\sin \sqrt[3]{x^2 + y^2}}{\sqrt[3]{ \left( x^2 + y^2 \right)^2}}
\end{equation}

\begin{equation}
\frac{\partial f}{\partial y} = \left( y + \cos \sqrt[3]{x^2 + y^2} \right)'_y = 1 -\frac{2y}{3} \frac{\sin \sqrt[3]{x^2 + y^2}}{\sqrt[3]{ \left( x^2 + y^2 \right)^2}}
\end{equation}
Терерь запишем полный дифференциал:

\begin{equation}
df =  -\frac{2x}{3} \frac{\sin \sqrt[3]{x^2 + y^2}}{\sqrt[3]{ \left( x^2 + y^2 \right)^2}} dx + \left( 1 -\frac{2y}{3} \frac{\sin \sqrt[3]{x^2 + y^2}}{\sqrt[3]{ \left( x^2 + y^2 \right)^2}}  \right) dy
\end{equation}

\newpage

\section{задание}
\textit{Найти производную данной функции в направлении данного вектора в заданной точке $M$. Пункт 8.}

\begin{equation}
f(x, y, z) = \exp(x + 2xy + 3xyz)
\end{equation}
по направлению внутренней нормали к поверхности $x^2 + y^2 + z^2 + 2z = 1$, $M \left( \frac{1}{2}, \frac{\sqrt{3}}{2}, 0 \right)$\\
\\
Найдем уравнение касательной плоскости к поверхности $x^2 + y^2 + z^2 + 2z = 1$ в точке $M \left( \frac{1}{2}, \frac{\sqrt{3}}{2}, 0 \right)$:

\begin{equation}
F'_x \left( M \right) \cdot (x - x_0) + F'_y \left( M \right) \cdot (y - y_0) + F'_z \left( M \right) \cdot (z - z_0) = 0
\end{equation}

\begin{equation}
F'_x = \left(  x^2 + y^2 + z^2 + 2z - 1 \right)'_x = 2x
\end{equation}

\begin{equation}
F'_y = \left(  x^2 + y^2 + z^2 + 2z - 1 \right)'_y = 2y
\end{equation}

\begin{equation}
F'_z = \left(  x^2 + y^2 + z^2 + 2z - 1 \right)'_z = 2z + 2
\end{equation}

\begin{equation}
F'_x  \left( M \right) = 2 \frac{1}{2} = 1
\end{equation}

\begin{equation}
F'_y  \left( M \right) = 2 \frac{\sqrt{3}}{2} = \sqrt{3}
\end{equation}

\begin{equation}
F'_z  \left( M \right) = 2
\end{equation}

Искомая плоскость:
\begin{equation}
x - \frac{1}{2}  +  \sqrt{3} \cdot (y - \frac{\sqrt{3}}{2}) + 2z= 0
\end{equation}

\begin{equation}
x  + \sqrt{3}y + 2z - 2= 0
\end{equation}
\\
Вектор нормали к плоскости $x  + \sqrt{3}y + 2z - 2= 0$ будет выглядеть $\vec{n} = \left( 1, \sqrt{3}, 2 \right)$. Поверхность  $x^2 + y^2 + z^2 + 2z = 1$ представляет собой сферу  $x^2 + y^2 + \left( z + 1\right)^2 = 2$ с центром $O = (0, 0, -1)$ и радиусом $R = \sqrt{2}$. Так как точкой начала вектора $\vec{n}$ является  $M \left( \frac{1}{2}, \frac{\sqrt{3}}{2}, 0 \right)$, а его координаты  $\vec{n} = \left( 1, \sqrt{3}, 2 \right)$, то точкой конца вектора будет $\left(  \frac{3}{2}, \frac{3 \sqrt{3}}{2}, 2 \right)$. То есть вектор $\vec{n} = \left( 1, \sqrt{3}, 2 \right)$ направлен наружу относительно поверхности.\\
\\
Внутренняя нормаль $\vec{n}_{in} = \left( -1, -\sqrt{3}, -2 \right)$. \\
Получим нормированный вектор, по направлению которого будем вычислять производную:
\begin{equation}
\left|\vec{n}_{in}  \right| = \sqrt{1 + 3 + 4} =  \sqrt{8} = 2\sqrt{2}
\end{equation}

\begin{equation}
l = \left( -\frac{1}{2\sqrt{2}}, - \frac{\sqrt{3}}{2\sqrt{2}}, - \frac{1}{\sqrt{2}} \right)
\end{equation}

Вычислим частные производные от $f(x, y, z) = \exp(x + 2xy + 3xyz)$ в точке $M \left( \frac{1}{2}, \frac{\sqrt{3}}{2}, 0 \right)$:

\begin{equation}
f'_x = \left(1 + 2y + 3yz \right) \exp(x + 2xy + 3xyz)
\end{equation}

\begin{equation}
f'_y  = \left( 2x + 3xz \right) \exp(x + 2xy + 3xyz)
\end{equation}

\begin{equation}
f'_z   = 3xy \exp(x + 2xy + 3xyz)
\end{equation}

\begin{equation}
f'_x  \left( M \right) =  \left(1 + 2 \frac{\sqrt{3}}{2} \right) \exp \left( \frac{1}{2} + 2 \frac{1}{2}\frac{\sqrt{3}}{2} \right) =
\left(1 + \sqrt{3} \right) \exp \left( \frac{\sqrt{3} + 1}{2} \right) 
\end{equation}

\begin{equation}
f'_y  \left( M \right) =  \left( 2\frac{1}{2}  \right) \exp \left(\frac{1}{2} + 2\frac{1}{2}\frac{\sqrt{3}}{2} \right) =
\exp \left(\frac{\sqrt{3} + 1}{2} \right)
\end{equation}

\begin{equation}
f'_z  \left( M \right) = 3\frac{1}{2}\frac{\sqrt{3}}{2} \exp \left(\frac{1}{2} + 2\frac{1}{2}\frac{\sqrt{3}}{2} \right) =
\frac{3 \sqrt{3}}{4} \exp \left(\frac{\sqrt{3} + 1}{2} \right)
\end{equation}
\\
Вычислим производную по направлению $l$ в точке $M$:

\begin{multline}
\frac{\partial f}{\partial l} = \left(\left(1 + \sqrt{3} \right) e^{ \frac{\sqrt{3} + 1}{2}} , e^{ \frac{\sqrt{3} + 1}{2}}, \frac{3 \sqrt{3}}{4} e^{ \frac{\sqrt{3} + 1}{2}  } \right)
\begin{pmatrix}
 -\frac{1}{2\sqrt{2}}\\
\\
- \frac{\sqrt{3}}{2\sqrt{2}}\\
\\
- \frac{1}{\sqrt{2}}
\end{pmatrix}
= \\
=   -\frac{1 + \sqrt{3}}{2\sqrt{2}} e^{ \frac{\sqrt{3} + 1}{2}} - \frac{\sqrt{3}}{2\sqrt{2}}  e^{ \frac{\sqrt{3} + 1}{2}}
 - \frac{1.5 \sqrt{3}}{2\sqrt{2}}  e^{ \frac{\sqrt{3} + 1}{2}}  =  - \frac{1 + 3.5 \sqrt{3}}{2\sqrt{2}}  e^{ \frac{\sqrt{3} + 1}{2}} 
\end{multline}

\textit{\textbf{Ответ:}} $ - \frac{1 + 3.5 \sqrt{3}}{2\sqrt{2}}  e^{ \frac{\sqrt{3} + 1}{2}} $





\newpage

\section{задание}
\textit{Произвести указанную замену в данном дифференциальном уравнении. Решить полученное дифференциальное уравнение в новых переменных. Показать, что найденное решение (в исходных переменных) удовлетворяет исходному уравнению. Пункт 3.}\\
\\
$u$ и $v$ -- новые независимые переменные, $w$ -- новая функция. $u=x+y, \, v = x - y, \, w+z = xy,$

\begin{equation}
\frac{\partial^2 z}{\partial x^2} + 2 \frac{\partial^2 z}{\partial x \partial y} + \frac{\partial^2 z}{\partial y^2} = 0
\end{equation}
 $z=z(x, y)$, $w=w(u, v)$, $u = u(x, y)$, $v = v(x, y)$

Для начала найдем все производные второго порядка функции $w$ по переменным $x$ и $y$:
\begin{equation}
w'_x = w'_u \cdot u'_x + w'_v \cdot v'_x = w'_u + w'_v
\end{equation}

% for x
\begin{equation}
w''_{ux}  = w''_{uu} \cdot u'_x + w''_{uv} \cdot v'_x = w''_{uu} + w''_{uv}
\end{equation}

\begin{equation}
w''_{vx}  = w''_{vu} \cdot u'_x + w''_{vv} \cdot v'_x = w''_{vu} + w''_{vv}
\end{equation}

Вторая производная по  $x$:
\begin{equation}
w''_{xx} = w''_{uu} + 2w''_{uv} + w''_{vv}
\end{equation}


\begin{equation}
w'_y = w'_u \cdot u'_y + w'_v \cdot v'_y = w'_u - w'_v
\end{equation}

\begin{equation}
w''_{uy}  = w''_{uu} \cdot u'_y + w''_{uv} \cdot v'_y = w''_{uu} - w''_{uv}
\end{equation}

\begin{equation}
w''_{vy}  = w''_{vu} \cdot u'_y + w''_{vv} \cdot v'_y = w''_{vu} - w''_{vv}
\end{equation}

 Вторая производная по $y$:
\begin{equation}
w''_{yy} = w''_{uu} - 2w''_{uv} + w''_{vv}
\end{equation}

Вторая производная по  $x$ и $y$:

\begin{equation}
w''_{yx} = w''_{xy} =  w''_{uu} -  w''_{vv}
\end{equation}
\\
Теперь выразим $z''_{xx}$,  $z''_{xy}$ и $z''_{yy}$ из $w + z = xy$:
\begin{equation}
z = xy - w
\end{equation}

\begin{equation}
z'_x = y - w'_x
\end{equation}

\begin{equation}
z'_y = x - w'_y
\end{equation}

\begin{equation}
z''_{xx} = - w''_{xx} = - ( w''_{uu} + 2w''_{uv} + w''_{vv})
\end{equation}


\begin{equation}
z''_{xy} = 1 - w''_{xy} = 1 - ( w''_{uu} -  w''_{vv})
\end{equation}

\begin{equation}
z''_{yy} = - w''_{yy} = -( w''_{uu} - 2w''_{uv} + w''_{vv})
\end{equation}
\\
Перепишем исходное уравнение в новых переменных и функции:

\begin{equation}
- ( w''_{uu} + 2w''_{uv} + w''_{vv}) + 2 (1 - ( w''_{uu} -  w''_{vv})) -( w''_{uu} - 2w''_{uv} + w''_{vv}) = 0
\end{equation}
\begin{equation}
- w''_{uu} - 2w''_{uv} - w''_{vv} + 2 -2  w''_{uu} +2  w''_{vv} - w''_{uu} + 2w''_{uv} - w''_{vv} = 0
\end{equation}
\begin{equation}
- w''_{uu}  + 2 -2  w''_{uu}  - w''_{uu}   = 0
\end{equation}
\begin{equation}
4 w''_{uu}  = 2
\end{equation}
\begin{equation}
 w''_{uu}  = \frac{1}{2}
\end{equation}
Решим полученное дифференциальное уравнение:

\begin{equation}
 \frac{d(w')}{du}  = \frac{1}{2}
\end{equation}

\begin{equation}
 d(w')  = \frac{1}{2} du
\end{equation}

\begin{equation}
 \int d(w')  = \int \frac{1}{2} du
\end{equation}

\begin{equation}
 w' =  \frac{u}{2} + C
\end{equation}

\begin{equation}
 dw = \left( \frac{u}{2} + C \right) du
\end{equation}

\begin{equation}
 \int dw =\int  \left( \frac{u}{2} + C \right) du
\end{equation}

\begin{equation}
 w =  \frac{u^2}{4} + Cu + A, \, A = const, \, C = const
\end{equation}

Решение в новых переменных:
\begin{equation}
 w =  \frac{u^2}{4} + Cu + A
\end{equation}

Выполним проверку: вернемся к старым переменным и подставим полученные выражения в исходное равенство:


\begin{equation}
u = x + y
\end{equation}

\begin{equation}
z = xy - w = xy -  \frac{u^2}{4} - Cu - A =  xy -  \frac{( x + y)^2}{4} - C( x + y) - A
\end{equation}

\begin{equation}
\frac{\partial^2 z}{\partial x^2} + 2 \frac{\partial^2 z}{\partial x \partial y} + \frac{\partial^2 z}{\partial y^2} = 0
\end{equation}

\begin{equation}
z'_x = \left(  xy -  \frac{( x + y)^2}{4} - C( x + y) - A \right)'_x = y - \frac{ x + y}{2}  -C = \frac{y}{2} - \frac{ x}{2} -C
\end{equation}

\begin{equation}
z'_y = \left(  xy -  \frac{( x + y)^2}{4} - C( x + y) - A \right)'_y = x - \frac{ x + y}{2}  -C = \frac{x}{2} - \frac{y}{2} -C
\end{equation}

\begin{equation}
z''_{xx} = \left(  \frac{y}{2} - \frac{ x}{2} -C \right)'_x =  - \frac{ 1}{2}
\end{equation}

\begin{equation}
z''_{xy} = \left(  \frac{y}{2} - \frac{ x}{2} -C \right)'_y =   \frac{ 1}{2}
\end{equation}

\begin{equation}
z''_{yy} = \left(  \frac{x}{2} - \frac{y}{2} - \right)'_y =  - \frac{ 1}{2}
\end{equation}

Подставим полученные выражения в исходное уравнение уравнение:

\begin{equation}
- \frac{ 1}{2} + 2\frac{ 1}{2} - \frac{ 1}{2} = 0
\end{equation}
Получили верное равенство.\\

\textit{\textbf{Ответ:}} $ w =  \frac{u^2}{4} + Cu + A $


\newpage

\section{задание}
\textit{Доказать, что уравнение $F(x,y)=0, \, F=(F_1, F_2)$ задает неявно дифференцируемое отображение $y=f(x), \, y=(y_1, y_2), \, x = (x_1, \, x_2)$ в окрестности точки $M(x_1, x_2, y_1, y_2)$. Найти производную этого отображения в точке $M$ (матрица Якоби) и одну (любую на выбор) из производных второго порядка $\frac{\partial f_i}{\partial x_j}$ в точке $M$. Пункт 3.}

\begin{equation}
\begin{cases}
F_1 (x, y) = x_1 + x_2 + y_1 + 2y_2 -5,\\
F_2 (x, y) = x^3_1 + x^2_2 +y^4_1+y^4_2-4,
\end{cases}
M(1, 1, 1, 1)
\end{equation}


\newpage

\section{задание}
\textit{С помощью метода Лагранжа исследовать функцию на условный экстремум при данном уравнении связи. Пункт 2.}

\begin{equation}
f(x, y, z) = x^2+2y^2+3z^2, \, x + 2y+3z=0, \, x^2 +y^2+z^2=1
\end{equation}
Запишем вид функции Лагранжа:
\begin{equation}
L = u(x, y, z) + \lambda_1 \phi_1 (x, y, z) + \lambda_2 \phi_2 (x, y, z)
\end{equation}
\begin{equation}
L =  x^2+2y^2+3z^2 + \lambda_1 ( x + 2y+3z) + \lambda_2 (x^2 +y^2+z^2 - 1)
\end{equation}

\begin{equation}
L'_x =  2x+ \lambda_1  + 2x \lambda_2
\end{equation}

\begin{equation}
L'_y = 4y + 2\lambda_1  + 2y \lambda_2 
\end{equation}

\begin{equation}
L'_z =  6z +3 \lambda_1 + 2z \lambda_2 
\end{equation}

\begin{equation}
\begin{cases}
L'_x = 0\\
L'_y = 0\\
L'_z = 0\\
\phi_1 (x, y, z) = 0\\
\phi_2 (x, y, z) = 0
\end{cases}
\end{equation}

\begin{equation}
\begin{cases}
 2x+ \lambda_1  + 2x \lambda_2 = 0\\
4y + 2\lambda_1  + 2y \lambda_2  = 0\\
6z +3 \lambda_1 + 2z \lambda_2  = 0\\
 x + 2y+3z = 0\\
x^2 +y^2+z^2 - 1 = 0
\end{cases}
\to
\begin{cases}
 \lambda_1  = -2x \lambda_2 -2x = -2x(\lambda_2 + 1)\\
2y + \lambda_1  + y \lambda_2  = 0\\
6z +3 \lambda_1 + 2z \lambda_2  = 0\\
 x + 2y+3z = 0\\
x^2 +y^2+z^2 - 1 = 0
\end{cases}
\end{equation}

\begin{equation}
\begin{cases}
 \lambda_1  = -2x \lambda_2 -2x = -2x(\lambda_2 + 1)\\
2y -2x(\lambda_2 + 1)  + y \lambda_2  = 0\\
3z -3x(\lambda_2 + 1) + z \lambda_2  = 0\\
 x + 2y+3z = 0\\
x^2 +y^2+z^2 - 1 = 0
\end{cases}
\end{equation}

\begin{equation}
2y -2x(\lambda_2 + 1)  + y \lambda_2  =0 \to  y(2+\lambda_2) = 2x(\lambda_2 + 1) \to y = \frac{2x(\lambda_2 + 1)}{2+\lambda_2}
\end{equation}

\begin{equation}
3z -3x(\lambda_2 + 1) + z \lambda_2  = 0 \to z(\lambda_2 + 3) = 3x(\lambda_2 + 1) \to z = \frac{ 3x(\lambda_2 + 1)}{\lambda_2 + 3}
\end{equation}

После подстановки $y$ и $z$, выраженных через $x$ и $\lambda_2$ в уравнения связи, получаем:
\begin{equation}
\begin{cases}
 x + \frac{4x(\lambda_2 + 1)}{2+\lambda_2}+\frac{9x(\lambda_2 + 1)}{\lambda_2 + 3} = 0\\
x^2 +\left(  \frac{2x(\lambda_2 + 1)}{2+\lambda_2} \right)^2+\left( \frac{ 3x(\lambda_2 + 1)}{\lambda_2 + 3} \right)^2 - 1 = 0
\end{cases}
\end{equation}
\\
Выразим $\lambda_2$ из первого уравнения:
\begin{equation}
 1 + \frac{4(\lambda_2 + 1)}{2+\lambda_2}+\frac{9(\lambda_2 + 1)}{\lambda_2 + 3} = 0
\end{equation}

\begin{equation}
 (2+\lambda_2)(\lambda_2 + 3) + 4(\lambda_2 + 1)(\lambda_2 + 3)+9(\lambda_2 + 1) (2+\lambda_2) = 0
\end{equation}
\begin{equation}
 6 + 5\lambda_2 + \lambda_2^2  + 4\lambda_2^2 + 16\lambda_2 + 12+9\lambda_2^2 +27\lambda_2 + 18 = 0
\end{equation}
\begin{equation}
14 \lambda_2^2 + 48  \lambda_2 +36  = 0
\end{equation}
\begin{equation}
7 \lambda_2^2 + 24  \lambda_2 +18  = 0
\end{equation}
 \\
Либо $x=0$, либо $\lambda_2= -\frac{12}{7} \pm \frac{3 \sqrt{2}}{7}$.
\\
\\
Рассмотрим $x = 0$:
\begin{equation}
\begin{cases}
 2y+3z = 0\\
y^2+z^2 - 1 = 0
\end{cases}
\end{equation}
\begin{equation}
y=-\frac{3z}{2}
\end{equation}
\\
\begin{equation}
z^2+\left( \frac{3z}{2} \right)^2 - 1 = 0
\end{equation}

\begin{equation}
\frac{13z^2}{4} = 1
\end{equation}

\begin{equation}
z = \pm \frac{2}{\sqrt{13}}
\end{equation}

\begin{equation}
z_1 = \frac{2}{\sqrt{13}} \to y_1 = -\frac{3}{\sqrt{13}}
\end{equation}

\begin{equation}
z_2 = -\frac{2}{\sqrt{13}} \to y_2 = \frac{3}{\sqrt{13}}
\end{equation}

\textbf{ДРУГОЙ СПОСОБ!?}\\
\\
Выразим $x = -2y-3z$ из первого уравнения связи и подставим во второе:

\begin{equation}
  (2y+3z)^2 +y^2+z^2=1
\end{equation}

\begin{equation}
  4y^2+12yz+9z^2 +y^2+z^2=1
\end{equation}


\end{document}













